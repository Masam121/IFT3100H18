\documentclass[12pt]{article}

\usepackage[english, frenchb]{babel}
\usepackage[utf8]{inputenc}
\usepackage[T1]{fontenc}

\usepackage{amsmath}
\usepackage{amsthm}
\usepackage{amssymb}
\usepackage{lmodern}

\newcommand{\state}{\noindent}
\newcommand{\substate}{\nolinebreak}

\begin{document}

\begin{center}
  \scriptsize
  \textbf{IFT3100H18\\}
  \textbf{Projet de session\\}
  \textbf{(énoncé partiel)\\}
\end{center}

\section*{Sommaire}

\state
L'objectif du projet de session est de développer une application qui permet de construire, éditer et rendre des scènes visuelles.\\

\state
Votre travail pratique sera principalement évalué selon votre implémentation de différents critères fonctionnels en lien avec la matière des modules du cours. \\

\state
Un total de 50 critères fonctionnels organisés en 10 catégories vous est proposée dans les pages suivantes. \\

\state
La première partie du projet de session (TP1) porte sur les critères fonctionnels des 5 premières catégories.
\substate
La seconde partie du projet de session (TP2) porte sur les critères fonctionnels des 5 dernières catégories. \\

\state
Il est recommandé d'implémenter environ 3 critères sur 5 par catégorie. \\

\state
Les choix liés aux technologies, au design logiciel et à la thématique du projet sont laissés à la préférence des membres de l'équipe. \\

\pagebreak

\section*{Document de design}

\subsection*{1. Sommaire}

\state
Description sommaire du projet (1 page).

\subsection*{2. Interactivité}

\state
Description de toutes les formes d'interactivité possibles ainsi que les principales entrées et sorties de l'application (1 page).

\subsection*{3. Technologie}

\state
Présentation des principaux outils technologiques utilisés pour le développement de l'application (de 1/2 page à 1 page).

\subsection*{4. Compilation}

\state
Présentation de la procédure pour réussir à compiler le projet sur un autre ordinateur (de 1/2 page à 1 page).

\subsection*{5. Architecture}

\state
Présentation de l'architecture logicielle de l'application sous forme textuelle et/ou de diagrammes (1 page).

\pagebreak

\subsection*{6. Fonctionnalités}

\state
Description des critères fonctionnels implémentés dans votre projet. \\

\state
Pour chacun des critères fonctionnels :

\begin{itemize}
\item[$\triangleright$]
Description sommaire de comment le critère fonctionnel a été réalisé dans votre projet.
\item[$\triangleright$]
Si pertinent, présenter une ou des images où le critère fonctionnel est mis en évidence.
\item[$\triangleright$]
Si pertinent, présenter l'essentiel du code qui a permis d'implémenter le critère fonctionnel. \\
\end{itemize}

\state
Prévoir environ 1-3 critères fonctionnels par page si seulement du texte ou 1-2 par page si accompagnés d'images et/ou d'extraits de code).

\subsection*{7. Ressources}

\state
Liste des ressources originales produites pour le projet et citation des références pour les ressources qui n'ont pas été produits par l'équipe (1 page).

\subsection*{8. Présentation}

\state
Présentation de l'équipe et de ses membres (1 page).

\pagebreak

\section{Image}

\subsection{Importation d'image}

\state
Il est possible d'importer des fichiers images et de les afficher dans une scène sous une forme ou une autre.

\qed

\subsection{Exportation d'image}

\state
Il est possible d'exporter des rendus d'une scène dans des fichiers images.

\qed

\subsection{Échantillonnage d'image}

\state
Il est possible de dessiner une image qui a été générée par un agencement d'échantillofaces inverséesns de pixels ou de blocs de pixels en provenance d'une autre image.

\qed

\subsection{Sélecteur de couleur}

\state
Il est possible de sélectionner une couleur parmi un ensemble de couleurs et de l'assigner à un élément visuel.

\qed

\subsection{Espace de couleur}

\state
Il est possible d'utiliser des couleurs en provenance d'un espace de couleur qui n'est pas du RGB.

\qed

\pagebreak

\section{Dessin vectoriel}

\subsection{Curseur dynamique}

\state
Il existe au moins 5 représentations visuelles différentes du curseur dessinées à partir de primitives vectorielles.

\qed

\subsection{Outils de dessin}

\state
Il est possible de modifier de manière interactive la valeur des outils de dessin vectoriel tel que l'épaisseur des lignes de contour, la couleur des lignes de contour, la couleur des zones de remplissage et la couleur d'arrière-plan de la scène.

\qed

\subsection{Primitives vectorielles}

\state
Il est possible de créer de manière interactive des instances d'au moins 5 types des primitives vectorielles parmi cet ensemble : point, ligne, carré, rectangle, triangle, quadrilatère, polygone régulier, polygone irrégulier, cercle, ellipse et arc.

\qed

\subsection{Forme vectorielle}

\state
Il est possible de créer des instances d'au moins 2 types de forme vectorielle composée d'un ensemble d'instances de primitives vectorielles.

\qed

\subsection{Interface}

\state
Un ou des éléments d'interface graphique offre de la rétroaction informative visuelle à l'utilisateur et des contrôles interactifs pour influencer les états de l'application.

\qed

\pagebreak

\section{Transformation}

\subsection{Graphe de scène}

\state
Tous les éléments visuels présents dans une scène sont organisés dans une ou des structures de données qui permettent l'ajout, la suppression et la sélection d'éléments.

\qed

\subsection{Sélection multiple}

\state
Il est possible de sélectionner plus d'une instance des éléments visuels
présents dans une scène et de modifier sur chaque élément de la sélection la valeur de certains attributs qu'ils ont en commun.

\qed

\subsection{Transformations interactives}

\state
Il est possible de modifier de manière interactive la translation, la rotation
et la proportion des éléments visuels présents dans une scène.

\qed


\subsection{Historique de transformation}

\state
Il est possible d'annuler ou de refaire (\textit{undo} / \textit{redo}) les dernières actions interactives qui ont un impact sur la transformation des éléments visuels présents dans une scène.


\qed

\subsection{Système de coordonnées}

\state
Il est possible de dessiner des éléments visuels qui sont transformés dans un système de coordonnées non-cartésien.

\qed


\pagebreak

\section{Géométrie}

\subsection{Boîte de délimitation}

\state
Une option permet de dessiner les arêtes d'une boîte d'une taille juste assez grande pour envelopper tous les sommets d'un modèle 3D pour chaque type de modèle qu'il est possible d'utiliser avec l'application.

\qed

\subsection{Primitives géométriques}

\state
Il est possible de dessiner au moins 2 types de primitives géométriques 3D générées à partir d'un algorithme qui n'utilise aucune données externes au programme.

\qed

\subsection{Modèle 3D}

\state
Il est possible de dessiner des instances d'au moins 2 types de modèles 3D importés à partir d'un fichier externe.

\qed

\subsection{Animation}

\state
Il est possible de dessiner au moins 1 modèle avec une animation de son maillage géométrique.

\qed

\subsection{Instanciation}

\state
Il est possible de dessiner de multiples instances d'un modèle qui ont des transformations différentes avec une seule commande d'affichage.

\qed

\pagebreak

\section{Texture}

\subsection{Mapping}

\state
Il existe au moins un modèle texturé avec des coordonnées de mapping adéquatement distribuées sur la surface du maillage géométrique, à l'exception de ceux-ci : triangle, plan, quad, cube.

\qed

\subsection{Composition}

\state
Il est possible de composer au moins 2 textures ensemble avec un mode de composition qui permet de voir la contribution de chaque texture sur au moins 1 élément visuel d'une scène.

\qed

\subsection{Traitement}

\state
Il est possible d'appliquer au moins 3 types d'algorithmes de traitement d'image qui affecte l'apparence d'une texture utilisée dans le rendu d'au moins 1 élément visuel d'une scène.

\qed

\subsection{Texture procédurale}

\state
Au moins 1 texture utilisée dans le rendu d'au moins 1 élément visuel d'une scène a été de manière procédurale par un algorithme.

\qed

\subsection{Cubemap}

\state
Il existe au moins 1 texture de type cubemap qui est utilisée dans le rendu d'au moins 1 élément visuel d'une scène.

\qed

\pagebreak

\section{Caméra}

\subsection{Point de vue}

\state
Il est possible de transformer une caméra par rapport à une ou des entités géométriques et l'utilisateur peut manipuler interactivement la caméra pour voir la position centrale de la sélection de différents points de vue, en plus de pouvoir s'en approcher et s'en éloigner.

\qed

\subsection{Mode de projection}

\state
Au moins deux type de mode de projection (ex: perspective et orthogonale) sont supportés par l'application et offrent à l'utilisateur un point de vue cohérent de la scène.

\qed

\subsection{Agencement}

\state
Il est possible de voir une scène de plusieurs points de vue différents, en pleine fenêtre ou avec des sous-fenêtres d'affichage dans la fenêtre principale de l'application où le rendu est fait en simultané dans chaque fenêtre.

\qed

\subsection{Occlusion}

\state
L'application utilise une technique d'occlusion autre que celles de base pour tenter de minimiser le nombre d'éléments du graphe de scène à rendre du point de vue d'une caméra (ex: BVH, quadtree, octree, etc).

\qed

\subsection{Portail}

\state
Un moins 1 effet visuel présent dans une scène est réalisé à partir d'une caméra qui rend dans une texture du contenu qui est ensuite affiché dans une scène rendue par une autre caméra (ex: portail, mirroir, effet de réflexion, écran de surveillance, etc). 

\qed

\pagebreak


\section{Illumination}

\subsection{Modèles d'illumination}

\state
Le rendu d'au moins 3 éléments visuels est fait à partir d'au moins 3 modèles d'illumination différents (ex: lambert, gouraud, phong, blinn-phong ou autre). 
\qed

\subsection{Matériaux}

\state
Au moins 3 éléments visuels d'une scène ont une surface avec un matériau sélectionné parmi un ensemble d'au moins 3 matériaux différents. 

\qed

\subsection{Types de lumière}

\state
Il est possible d'avoir au moins une instance de 4 types de lumières différents (ambiante, directionnelle, ponctuelle, projecteur ou autre).

\qed

\subsection{Volume de lumière}

\state
Il existe au moins 1 type de lumière qui permet d'émettre de la lumière seulement à l'intérieur d'un volume délimité par un modèle géométrique.

\qed

\subsection{Lumières multiples}

\state
Possibilité d'avoir au moins 4 différentes instances de lumière dynamique dont la couleur, la position et l'atténuation sont prises en compte lors des calculs d'illumination d'au moins 1 type de matériau.

\qed

\pagebreak

\section{Lancer de rayon}

\subsection{Intersection}

\state
L'application est capable de calculer l'intersection entre un rayon et au moins 2 types de primitives géométriques.

\qed

\subsection{Réflexion}

\state
Une technique de rendu inspirée des principes du lancer de rayon est utilisée pour rendre au moins 1 effet de réflexion (ex: une surface miroir).

\qed

\subsection{Réfraction}

\state
Une technique de rendu inspirée des principes du lancer de rayon est utilisée pour rendre au moins 1 effet de réfraction. (ex: une surface en verre)

\qed

\subsection{Ombrage}

\state
Une technique de rendu inspirée des principes du lancer de rayon est utilisée pour rendre au moins 1 effet d'ombrage.

\qed

\subsection{Illumination globale}

\state
Une technique de rendu inspirée des principes de l'illumination globale est utilisée pour rendre au moins 1 scène.

\qed

\pagebreak

\section{Topologie}

\subsection{Courbe cubique}

\state
Il est possible de rendre au moins 2 types de courbes cubiques avec 4 points de contrôle, comme par exemple une courbe de Hermite ou une courbe de Bézier cubique.

\qed

\subsection{Courbe paramétrique}

\state
Il est possible de rendre au moins 1 type de courbe paramétrique avec plus de 4 points de contrôle, comme par exemple une spline de Bézier ou de Catmull-Rom.

\qed

\subsection{Surface paramétrique}

\state
Il est possible de rendre au moins 1 surface paramétrique, comme par exemple une surface de Bézier bicubique ou une surface de Coons.

\qed

\subsection{Shader de tesselation}

\state
Un shader de tesselation permet de sous-diviser la géométrie d'au moins 1 modèle, comme par exemple un plan ou une surface paramétrique.

\qed

\subsection{Triangulation}

\state
Un algorithme permet de générer au moins 1 maillage triangulaire à partir d'un ensemble de sommets, comme par exemple une triangulation de Delaunay ou un diagramme de Voronoï.

\pagebreak


\pagebreak

\section{Techniques de rendu}

\subsection{Rendu en différé}

\state

\qed

\subsection{Effet en pleine fenêtre}

\state

\qed

\subsection{Normal mapping}

\state

\qed

\subsection{BRDF}

\state

\qed

\subsection{Style libre}

\state

\qed

\pagebreak

\section*{Évaluation}

\state
Voici les informations par rapport à l'évaluation du projet de session :

\begin{itemize}
\item[$\triangleright$] Le projet de session est séparé en deux livrables : TP1 et TP2.
\item[$\triangleright$] Le TP1 et TP2 ont une valeur de 20 \% de la session chacun.
\item[$\triangleright$] Ils seront corrigés sur 100 \%.
\item[$\triangleright$] 10 \% sera pour le document de design
\item[$\triangleright$] 90 \% sera pour les critères fonctionnels.
\item[$\triangleright$] Chaque critère fonctionnel sera évalué de 0 à 3 points. \\
$\bigstar\bigstar\bigstar$ au-delà des attentes\\
$\bigstar\bigstar$ au niveau des attentes\\
$\bigstar$ en bas des attentes
\item[$\triangleright$] Une implémentation de bonne qualité d'un des critères fonctionnels vaudra généralement 2 points.
\item[$\triangleright$] Tous les critères fonctionnels sont optionnels.
\item[$\triangleright$] Le maximums de point pour les critères fonctionnels est de 40 points. \\
\end{itemize}

\pagebreak

\section*{Livraison}

\state
Voici les informations par rapport à la livraison du travail pratique :

\begin{itemize}
\item[$\triangleright$]
La livraison du travail pratique doit se faire dans la boîte de dépôt de l'équipe sur le site web du cours. Aucun autre moyen de livraison ne sera accepté.
\item[$\triangleright$]
Tout le contenu du livrable doit se trouver à l'intérieur d'un répertoire.
\item[$\triangleright$]
Ce répertoire doit contenir au minimum le code source complet du projet ainsi que toutes les ressources nécessaires à sa compilation et son exécution, au moins une version compilée ainsi qu'une version publiée en format \textbf{.pdf} du document de design.
\item[$\triangleright$] Ce répertoire doit être la racine du projet, mais vous pouvez organiser son contenu comme bon vous semble avec d'autres sous-répertoires (ex: \textit{/doc}, \textit{/src}, \textit{/build}, \textit{/ref}, etc).
\item[$\triangleright$] Ce répertoire doit être compressé dans une archive de format \textbf{.zip} avant la livraison dans la boîte de dépôt.
\item[$\triangleright$]
La limite de poids permise pour l'archive du projet est la limite permise par la boîte de dépôt (250 Mo).
\item[$\triangleright$]
Le système de boîte de dépôt permet de faire plusieurs livraisons, mais seulement la dernière version livrée sera évaluée.
\item[$\triangleright$]
Prière d'éviter les espaces, les caractères accentués et les caractères spéciaux dans tous les noms de fichiers et de répertoire.
\item[$\triangleright$]
Il est fortement recommandé de tester votre projet sur un autre ordinateur avant la livraison pour confirmer que tout fonctionne comme prévu.
\item[$\triangleright$]
La date de livraison du TP1 est le dimanche \textbf{11 mars} avant minuit.
\item[$\triangleright$]
La date de livraison du TP2 est le dimanche \textbf{30 avril} avant minuit.
\item[$\triangleright$]
La pénalité de retard est de \textbf{0.5\%} par heure.
\end{itemize}

\pagebreak

\section*{Prix Pierre-Ardouin}

\state
Le projet de session est éligible au Prix Pierre-Ardouin, dont voici les principales informations : \\

\state
\textit{Depuis l'automne 2013, le Département d'informatique et de génie logiciel a mis en place un concours récompensant l'équipe qui aura produit le meilleur TP/projet dans le cadre d'un cours. Ces travaux de session ont l'envergure d'un mini-projet qui est admissible par rapport aux normes fixées par le Département.} \\

\state
\textit{À la suite des évaluations des travaux, l'enseignant du cours détermine l'équipe gagnante ; chaque membre de l'équipe gagnante reçoit alors une bourse de 50\$ ainsi qu'une attestation remises par le Département.
De plus, le Département d'informatique et de génie logiciel a mis en place une bourse Élite, appelée bourse « Pierre Ardouin », qui vise à récompenser le meilleur projet de session, tous cours confondus.} \\

\state
\textit{Deux principaux critères guident le choix des évaluateurs dans l'identification du lauréat : l'excellence du travail (par rapport à ce qui est demandé dans l'énoncé) et l'aspect créativité/innovation. Il est actuellement prévu une bourse de 200\$ pour récompenser chaque membre de l'équipe « élite » gagnante (pour un maximum de 1000\$ pour toute l'équipe).} \\

\state
\textit{Aussi, le Département veille à publier l'information sur un site Web dédié : http://www.ift.ulaval.ca/vie-etudiante/prix-pierre-ardouin.} \\

\state
\textit{À la deuxième moitié du mois de mai de chaque année universitaire, le Département organise une cérémonie pour honorer les finalistes et le lauréat du prix «Pierre Ardouin» des sessions d'automne et d'hiver, et leur remettre une attestation.»} \\

\state
Si vous souhaitez participer au concours, vous devez présenter votre projet sur
le forum 'Présentation des projets' avant le 1 mai.

\end{document}