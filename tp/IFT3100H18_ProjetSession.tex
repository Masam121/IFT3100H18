\documentclass[12pt]{article}

\usepackage[english, frenchb]{babel}
\usepackage[utf8]{inputenc}
\usepackage[T1]{fontenc}

\usepackage{amsmath}
\usepackage{amsthm}
\usepackage{amssymb}
\usepackage{lmodern}

\newcommand{\state}{\noindent}
\newcommand{\substate}{\nolinebreak}

\begin{document}

\begin{center}
  \scriptsize
  \textbf{IFT3100H18\\}
  \textbf{Projet de session\\}
\end{center}

\section*{Sommaire}

\state
L'objectif du projet de session est de développer une application qui permet de construire, éditer et rendre des scènes visuelles.\\

\state
Votre travail pratique sera principalement évalué selon votre implémentation de différents critères fonctionnels en lien avec la matière des modules du cours. \\

\state
Un total de 50 critères fonctionnels organisés en 10 catégories vous est proposée dans les pages suivantes. \\

\state
La première partie du projet de session (TP1) porte sur les critères fonctionnels des 5 premières catégories.
\substate
La seconde partie du projet de session (TP2) porte sur les critères fonctionnels des 5 dernières catégories. \\

\state
Il est recommandé d'implémenter environ 3 critères sur 5 par catégorie. \\

\state
Les choix liés aux technologies, au design logiciel et à la thématique du projet sont laissés à la préférence des membres de l'équipe. \\

\pagebreak

\section*{Document de design}

\subsection*{1. Sommaire}

\subsection*{2. Interactivité}

\subsection*{3. Technologie}

\subsection*{4. Compilation}

\subsection*{5. Architecture}

\subsection*{6. Fonctionnalités}

\subsection*{7. Ressources}

\subsection*{8. Présentation}

\pagebreak

\section{Image}

\subsection{Importation d'image}

\state
Il est possible d'importer des fichiers images et de les afficher dans une scène sous une forme ou une autre.

\qed

\subsection{Exportation d'image}

\state
Il est possible d'exporter des rendus d'une scène dans des fichiers images.

\qed

\subsection{Échantillonnage d'image}

\state
Il est possible de dessiner une image qui a été générée par un agencement d'échantillons de pixels ou de blocs de pixels en provenance d'une autre image.

\qed

\subsection{Sélecteur de couleur}

\state
Il est possible de sélectionner une couleur parmi un ensemble de couleurs et de l'assigner à un élément visuel.

\qed

\subsection{Espace de couleur}

\state
Il est possible d'utiliser des couleurs en provenance d'un espace de couleur qui n'est pas du RGB.

\qed

\pagebreak

\section{Dessin vectoriel}

\subsection{Curseur dynamique}

\state
Il existe au moins 5 représentations visuelles différentes du curseur dessinées à partir de primitives vectorielles.

\qed

\subsection{Outils de dessin}

\state
Il est possible de modifier de manière interactive la valeur des outils de dessin vectoriel tel que l'épaisseur des lignes de contour, la couleur des lignes de contour, la couleur des zones de remplissage et la couleur d'arrière-plan de la scène.

\qed

\subsection{Primitives vectorielles}

\state
Il est possible de créer de manière interactive des instances d'au moins 5 types des primitives vectorielles parmi cet ensemble : point, ligne, carré, rectangle, triangle, quadrilatère, polygone régulier, polygone irrégulier, cercle, ellipse et arc.

\qed

\subsection{Forme vectorielle}

\state
Il est possible de créer des instances d'au moins 2 types de forme vectorielle composée d'un ensemble d'instances de primitives vectorielles.

\qed

\subsection{Interface}

\state
Un ou des éléments d'interface graphique offre de la rétroaction informative visuelle à l'utilisateur et des contrôles interactifs pour influencer les états de l'application.

\qed

\pagebreak

\section{Transformation}

\subsection{Graphe de scène}

\state
Tous les éléments visuels présents dans une scène sont organisés dans une ou des structures de données qui permettent l'ajout, la suppression et la sélection d'éléments.

\qed

\subsection{Sélection multiple}

\state
Il est possible de sélectionner plus d'une instance des éléments visuels
présents dans une scène et de modifier sur chaque élément de la sélection la valeur de certains attributs qu'ils ont en commun.

\qed

\subsection{Transformations interactives}

\state
Il est possible de modifier de manière interactive la translation, la rotation
et la proportion des éléments visuels présents dans une scène.

\qed


\subsection{Historique de transformation}

\state
Il est possible d'annuler ou de refaire (\textit{undo} / \textit{redo}) les dernières actions interactives qui ont un impact sur la transformation des éléments visuels présents dans une scène.


\qed

\subsection{Système de coordonnées}

\state
Il est possible de dessiner des éléments visuels qui sont transformés dans un système de coordonnées non-cartésien.

\qed


\pagebreak

\section{Géométrie}

\subsection{Nuage de points}

\state

\qed

\subsection{Primitives géométriques}

\state

\qed

\subsection{Modèle 3D}

\state

\qed

\subsection{Instanciation}

\state

\qed

\subsection{Animation}

\state

\qed

\pagebreak

\section{Texture}

\subsection{Mapping}

\state

\qed

\subsection{Composition}

\state

\qed

\subsection{Traitement}

\state

\qed

\subsection{Texture procédurale}

\state

\qed

\subsection{Cubemap}

\state

\qed

\pagebreak

\section{Caméra}

\subsection{Point de vue}

\state

\qed

\subsection{Mode de projection}

\state

\qed

\subsection{Agencement}

\state

\qed

\subsection{Occlusion}

\state

\qed

\subsection{Portail}

\state

\qed

\pagebreak


\section{Illumination}

\subsection{Modèles d'illumination}

\state

\qed

\subsection{Matériaux}

\state

\qed

\subsection{Types de lumière}

\state

\qed

\subsection{Volume de lumière}

\state

\qed

\subsection{Lumières multiples}

\state

\qed

\pagebreak

\section{Lancer de rayon}

\subsection{Intersection}

\state

\qed

\subsection{Réflexion}

\state

\qed

\subsection{Réfraction}

\state

\qed

\subsection{Ombrage}

\state

\qed

\subsection{Illumination globale}

\state

\qed

\pagebreak

\section{Topologie}

\subsection{Courbe cubique}

\state

\qed

\subsection{Courbe paramétrique}

\state

\qed

\subsection{Surface paramétrique}

\state

\qed

\subsection{Triangulation}

\state

\qed

\subsection{Shader de tesselation}

\state

\qed


\pagebreak

\section{Techniques de rendu}

\subsection{Rendu en différé}

\state

\qed

\subsection{Effet en pleine fenêtre}

\state

\qed

\subsection{Normal mapping}

\state

\qed

\subsection{BRDF}

\state

\qed

\subsection{Style libre}

\state

\qed

\pagebreak

\section*{Évaluation}

\pagebreak

\section*{Livraison}

\pagebreak

\end{document}